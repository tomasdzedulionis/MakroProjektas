\PassOptionsToPackage{unicode=true}{hyperref} % options for packages loaded elsewhere
\PassOptionsToPackage{hyphens}{url}
%
\documentclass[
]{article}
\usepackage{lmodern}
\usepackage{amssymb,amsmath}
\usepackage{ifxetex,ifluatex}
\ifnum 0\ifxetex 1\fi\ifluatex 1\fi=0 % if pdftex
  \usepackage[T1]{fontenc}
  \usepackage[utf8]{inputenc}
  \usepackage{textcomp} % provides euro and other symbols
\else % if luatex or xelatex
  \usepackage{unicode-math}
  \defaultfontfeatures{Scale=MatchLowercase}
  \defaultfontfeatures[\rmfamily]{Ligatures=TeX,Scale=1}
\fi
% use upquote if available, for straight quotes in verbatim environments
\IfFileExists{upquote.sty}{\usepackage{upquote}}{}
\IfFileExists{microtype.sty}{% use microtype if available
  \usepackage[]{microtype}
  \UseMicrotypeSet[protrusion]{basicmath} % disable protrusion for tt fonts
}{}
\makeatletter
\@ifundefined{KOMAClassName}{% if non-KOMA class
  \IfFileExists{parskip.sty}{%
    \usepackage{parskip}
  }{% else
    \setlength{\parindent}{0pt}
    \setlength{\parskip}{6pt plus 2pt minus 1pt}}
}{% if KOMA class
  \KOMAoptions{parskip=half}}
\makeatother
\usepackage{xcolor}
\IfFileExists{xurl.sty}{\usepackage{xurl}}{} % add URL line breaks if available
\IfFileExists{bookmark.sty}{\usepackage{bookmark}}{\usepackage{hyperref}}
\hypersetup{
  pdftitle={Antiinfliacinės priemonės ir jų efektyvumas},
  pdfborder={0 0 0},
  breaklinks=true}
\urlstyle{same}  % don't use monospace font for urls
\usepackage[left=3cm,right=3cm,top=2cm,bottom=2cm]{geometry}
\usepackage{graphicx,grffile}
\makeatletter
\def\maxwidth{\ifdim\Gin@nat@width>\linewidth\linewidth\else\Gin@nat@width\fi}
\def\maxheight{\ifdim\Gin@nat@height>\textheight\textheight\else\Gin@nat@height\fi}
\makeatother
% Scale images if necessary, so that they will not overflow the page
% margins by default, and it is still possible to overwrite the defaults
% using explicit options in \includegraphics[width, height, ...]{}
\setkeys{Gin}{width=\maxwidth,height=\maxheight,keepaspectratio}
\setlength{\emergencystretch}{3em}  % prevent overfull lines
\providecommand{\tightlist}{%
  \setlength{\itemsep}{0pt}\setlength{\parskip}{0pt}}
\setcounter{secnumdepth}{5}
% Redefines (sub)paragraphs to behave more like sections
\ifx\paragraph\undefined\else
  \let\oldparagraph\paragraph
  \renewcommand{\paragraph}[1]{\oldparagraph{#1}\mbox{}}
\fi
\ifx\subparagraph\undefined\else
  \let\oldsubparagraph\subparagraph
  \renewcommand{\subparagraph}[1]{\oldsubparagraph{#1}\mbox{}}
\fi

% set default figure placement to htbp
\makeatletter
\def\fps@figure{htbp}
\makeatother

\usepackage[utf8]{inputenc}
\usepackage[L7x]{fontenc}
\usepackage[lithuanian]{babel}
\usepackage{float}
\usepackage{setspace}
\onehalfspacing

\usepackage{graphicx}
\usepackage{booktabs}

\usepackage{hyperref}
\hypersetup{colorlinks=true,
linkcolor=black,
filecolor=black,
urlcolor=black,
citecolor=red}

\usepackage{float}
\let\origfigure\figure
\let\endorigfigure\endfigure
\renewenvironment{figure}[1][2] {
    \expandafter\origfigure\expandafter[H]
} {
    \endorigfigure
}
% https://github.com/rstudio/rmarkdown/issues/337
\let\rmarkdownfootnote\footnote%
\def\footnote{\protect\rmarkdownfootnote}

% https://github.com/rstudio/rmarkdown/pull/252
\usepackage{titling}
\setlength{\droptitle}{-2em}

\pretitle{\vspace{\droptitle}\centering\huge}
\posttitle{\par}

\preauthor{\centering\large\emph}
\postauthor{\par}

\predate{\centering\large\emph}
\postdate{\par}
\usepackage{booktabs}
\usepackage{longtable}
\usepackage{array}
\usepackage{multirow}
\usepackage{wrapfig}
\usepackage{float}
\usepackage{colortbl}
\usepackage{pdflscape}
\usepackage{tabu}
\usepackage{threeparttable}
\usepackage{threeparttablex}
\usepackage[normalem]{ulem}
\usepackage{makecell}
\usepackage{xcolor}

\title{Antiinfliacinės priemonės ir jų efektyvumas}
\author{Tomas Dzedulionis ir Dominykas Dzingelevičius\\
Vilniaus universitetas Ekonomikos ir verslo administravimo fakultetas}
\date{01/06/2020}

\begin{document}
\maketitle

{
\setcounter{tocdepth}{3}
\tableofcontents
}
\newpage

\hypertarget{ux12fvadas}{%
\section{Įvadas}\label{ux12fvadas}}

\hypertarget{darbo-aktualumas}{%
\subsection{Darbo aktualumas}\label{darbo-aktualumas}}

\hypertarget{darbo-uux17edaviniai}{%
\subsection{Darbo uždaviniai}\label{darbo-uux17edaviniai}}

\newpage

\hypertarget{istoriniai-poux17eiux16briai.}{%
\section{Istoriniai požiūriai.}\label{istoriniai-poux17eiux16briai.}}

\hypertarget{klasikai}{%
\subsection{Klasikai}\label{klasikai}}

Klasikinė ekonomikos teorija, atsiradusi XVIIIa. sulig Adamo Smitho
„Tautų Turto`` pasirodymu, ir jos pasekėjai pagrindine infliacijos
atsiradimo priežastimi laikė pernelyg didelį pinigų kiekį ekonomikoje.
Dėl šios priežasties klasikų teorija dažnu atveju yra vadinama kiekybine
pinigų teorija (Ireland 2014). Savo darbuose klasikai bendrojo kainų
lygio svyravimus aiškino pasitelkdami fundamentalią pinigų paklausos ir
pinigų pasiūlos sąveika - padidėjusi pinigų pasiūla didina bendrąjį
kainų lygį, o tai sumažina piniginio vieneto perkamąją galią bei sukelia
infliaciją. Ši teorija leidžia į infliacijos problemą žvelgti tiesiausiu
kampu - ignoruojant tokias dedamąsias kaip palūkanų lygis ar nedarbas.
Klasikų suformuotos teorijos išvados, kaip pagrindinį infliacijos
suvaldymo įrankį, iškelia centrinių bankų atsakomybę - pinigų pasiūlos
mažinimą. Suformuota teorija ir jos kūrėjai tikėjo, jog pagrindinis
ekonomikos valdymas vyksta reguliuojant pinigų kiekį, o pagrindiniu
antiinfliaciniu mechanizmu laikė ``šalto dušo'' ekonomikai taikymą, kai
drastiškai sumažinama pinigų pasiūla, apie tai pranešama iš anksto ir
taip sumažinami infliaciniai lūkesčiai (Ball 2017). Sumažėjusi pinigų
paklausa padidins palūkanų normą ir taip sumažins išlaidavimą ir
skolinimąsį bei sumažins konkurenciją dėl resursų, o stabilus pinigų
pasiūlos didinimas tokiu pat lygiu, kokiu auga ir ekonomika, užtikrins
stabilų kainų lygį. Šią infliacinę teoriją atspindi ir pavyzdžiui JAV
Federalinio rezervų banko tikslas - naudojant pinigų kiekio reguliacijos
mechanizmus užtikrinti kainų stabilumą ir visišką užimtumą, tačiau
klasikų siūlytos šalto dušo priemonės susilaukė kritikos ir tolimesnių
tyrimų.

\hypertarget{keinsistai.}{%
\subsection{Keinsistai.}\label{keinsistai.}}

J.M.Keynes'as ir jo pasekėjai kritikavo klasikų požiūrį bei teigė, jog
monetarinė politika turi įtakos infliacijai, tačiau tik kaip viena iš
kompleksinės priežasčių grandinės dedamųjų dalių. Savo gyvenimo metais
jis padėjo pagrindus keinsistinei ekonomikai ir visuminės paklausos
(angl. demand pull inflation) ir visuminės pasiūlos (angl. cost push
inflation) sukeltos infliacijos aiškinimui. Savo ankstyvuosiuose
darbuose Keynes'as rašė, jog valiutos nuvertėjimas ir kainų kilimas nėra
naudingas valstybei ir tarptautinei prekybai, kadangi neigiamai
atsiliepia didžiajai visuomenės daliai bei sutiko su Leninu ir
infliaciją įvardino kaip vieną iš didžiausių grėsmių kapitalistinei
sistemai (Humphrey 1981). Keinsistinių pažiūrų ekonomika teigia, jog
infliacija atsiranda ekonomikoje susidarius tokiai konjunktūrai, kai
žmonių išlaidos yra didesnės nei gėrybių pasiūlą rinkoje, todėl tam, kad
rinka išsivalytų, kainos turi kilti (Weidenaar 1979). XXa. antrajame
dešimtmetyje J.M.Keynes'as buvo monetaristų pusėje ir tikėjo, jog pinigų
pasiūlos ir palūkanų keitimo politika yra geriausias būdas užtikrinti
kainų stabilumą, o taip pat, kylant infliacijos grėsmei, Keynes'as siūlė
ir drastiškas šalto dušo priemones (O'Connell 2016). Po Pirmojo
pasaulinio karo Keynes'as iškėlė idėją, jog centriniai bankai turėtų
būti nepriklausomi nuo vyriausybės ir nefinansuoti valstybės išlaidų,
taip pat, matydamas, jog nepaisant monetarinio skatinimo ir žemos
palūkanų normos Britanija kentėjo nuo aukšto nedarbo lygio ir bendro
ekonomikos strigimo, Keynes'as perėjo į fiskalinės politikos pusę ir
pristatė visuminės paklausos idėja. Laikui bėgant, Keinso pasekėjai
išplėtojo jo idėjas, jomis remiantis sukūrė visuminės paklausos ir
visuminės paklausos modelį, o keinsistinė ekonomika pabrėžia valstybes
galią kovojant su infliacija ir pagrindine infliacijos suvaldymo
priemone laiko visuminės paklausos ir pasiūlos korekcijas naudojant
restrikcinę fiskalinę politiką ir atmeta monetaristų siūlytą šalto dušo
metodą, kuris, anot jų, nėra tvarus (O'Connell 2016).

\hypertarget{aukos-santykis}{%
\subsection{Aukos santykis}\label{aukos-santykis}}

Klausimas, kuris infliacijos mažinimo būdas - klasikų siūlytas šaltas
dušas ir staigios bei drastiškos priemonės ar nuosaikus infliacijos
mažinimas - yra geresnis, ginčų objektu buvo ganėtinai ilgą laiką. 1994m
Amerikiečių ekonomistas Laurence M. Ball publikacijoje „\emph{What
Determines the Sacrifice Ratio}`` empirinių stebėjimų pagalba surinko
duomenis ir apskaičiavo abiejų šių infliacijos mažinimo būdų aukos
santykius. Jis teigė, jog moderniosiose ekonomikose disinfliacija yra
pagrindinė recesijų priežastis. Aukos santykis - tai prarastos gamybos
kiekis, kiekvienam infliacijos tempo sumažėjimo procentiniam punktui
(Ball 1994).\\
Disinfliacija visais atvejais kainuoja prarastą gamybos kiekį, tačiau
skirtingi požiūriai į infliacijos mažinimą pateikia skirtingus
privalumus ir trūkumus. Nuosaikių priemonių šalininkai teigia, jog toks
mažinimo būdas yra mažiau žalingas, kadangi atlyginimai ir kainų lygis
yra inertiški, todėl jiems prireikia laiko prisitaikyti prie monetarinės
restrikcijos (Taylor 1982). Staigių priemonių ir ``šalto dušo''
šalininkai teigė, jog greitos ir ryžtingos priemonės kelia pasitikėjimą
ir efektyviai bei greitai sumažina infliacinius lūkesčius taip pat
panaikina ``meniu kaštus''. Jie teigė, kad nuosaikios mažinimo priemonės
gali reikšmingai nepaveikti infliacinių lūkesčių ir paskatinti
visuomenės spekuliacijas apie infliacijos kilimą ateityje (Sargent and
others 1981). Tyrimo metu gauti vidutiniai aukos santykiai matomi
lentelėje (žr. 1-ą lentelę), jie varijuoja nuo \emph{0.22} Prancūzijoje
iki \emph{2.52} Vokietijoje. Tyrinėjant įvairius atvejus, nustatyta, jog
mažesni produkcijos nuostoliai patirti staigaus infliacijos mažinimo
priemonių ir lankstaus darbo užmokesčio atveju. Autorius pažymi, jog
tokie rezultatai galioja tik tuomet, kai priemones vertiname per
prarastos produkcijos prizmę, tačiau nevertiname bendros gerovės,
kadangi nuosaikus mažinimo politika gerovės nuostolius sugebėtų
paskirstyti ilgesniame laikotarpyje taip sumažinant momentinį šoką, o
vyriausybės visų pirma turėtų rūpintis didesniu darbo užmokesčio
lankstumu ir mažinti terminuotų darbo kontraktų trukmę (Ball 1994).
Tyrimo rezultatai susilaukė kritikos dėl to, jog matavimai neatsižvelgia
į veiksnius, kurie veikia tiek infliaciją, tiek gamybą, kaip pavyzdžiui
pasiūlos šokus, kas reiškia, jog galutinis vertinimas nėra visiškai
tikslus ir turėtų būti vertinamas kaip vienas iš galimų variantų.

\begin{table}[H]

\caption{\label{tab:unnamed-chunk-2}„Average sacrifice ratios by country“ (Ball 1994)}
\centering
\begin{tabular}[t]{l|l}
\hline
Valstybė & Aukos\_Santykis\\
\hline
Australija & 0.32\\
\hline
Austrija & 0.47\\
\hline
Belgija & 0.98\\
\hline
Kanada & 1.20\\
\hline
Danija & 0.56\\
\hline
Suomija & 0.72\\
\hline
Prancūzija & 0.22\\
\hline
Vokietija & 2.52\\
\hline
Airija & 0.72\\
\hline
Italja & 1.48\\
\hline
Japonija & -0.23\\
\hline
Liuksemburgas & 0.53\\
\hline
Nyderlandai & 0.31\\
\hline
Naujoji Zelandija & 0.53\\
\hline
Ispanija & 0.90\\
\hline
Švedija & 0.45\\
\hline
Šveicarija & 0.86\\
\hline
Jungtinė Karalystė & 0.68\\
\hline
JAV & 2.30\\
\hline
\end{tabular}
\end{table}

\hypertarget{fiskalinux117s-priemonux117s}{%
\section{Fiskalinės priemonės}\label{fiskalinux117s-priemonux117s}}

Valstybės dažnu atveju naudoja fiskalines priemones ekonomikos
stabilumui užtikrinti. Pagrindiniai du fiskalinės politikos komponentai
yra vyriausybės pajamos, surenkamos mokesčių pagalba ar skolinantis, ir
išlaidos. Egzistuoja ir skolinimosi bei skolos valdymo politika, tačiau
ji glaudžiai susijusi su centrinio banko vykdoma monetarine politika ir
dažnu atveju yra išskiriama kaip atskira nuo fiskalinės ir monetarinės
politikų stabilizavimo rūšis (Husain 2015).Infliacija kontroliuojama
mažinant išlaidas, didinant pajamas arba naudojant abi priemones vienu
metu. Pagrindinis tokių priemonių tikslas - ekonomikos stabilumas t.y.
aukštas užimtumo ir stabilus kainų lygiai. Nepaisant fiskalinės
politikos trūkumų, ji taip pat turi ir minusų, tokių kaip mažas poveikis
pasiūlos sukeltai infliacijai bei dar didesnių problemų sukėlimas, kai
fiksuojama didelė infliacija ir neigiamas produkcijos augimas.

\hypertarget{vyriausybux117s-iux161laidos}{%
\subsection{Vyriausybės išlaidos}\label{vyriausybux117s-iux161laidos}}

Vyriausybė, siekdama kontroliuoti visuminę paklausą ir infliaciją, gali
mažinti savo išlaidas ir taip atsverti privataus vartojimo padidėjimą.
Valstybės išlaidos turėtų būti sąmoningai mažinamos siekiant sulaikyti
naujų pinigų tiekimą, tačiau jos nėra labai elastingos ir lengvai
prisitaikančios prie pokyčių, reikalingų infliacijai valdyti (Husain
2015). Taip pat, valstybė privalo užtikrinti socialinę ir sveikatos
apsaugą, sienų gynybą, viešojo sektoriaus išlaikymą, todėl egzistuoja
tam tikri limitai, žemiau kurių išlaidų mažinti negalima, o tai suveikia
kaip ribojantis šios priemonės veiksnys. Taip pat, mažinamos išlaidos
atima politinius dividendus iš sprendimus priimančių politikų, todėl
mažinamos išlaidos nėra dažnai naudojama priemonė. Teigiamas sąryšis
tarp vyriausybės išlaidų ir infliacijos atsispindi ir grafike (žr. 1
pav.). Grafike analizuojami JAV Federalinio rezervo banko duomenys -
metinė infliacija ir kasmetinis vyriausybės išlaidų pokytis.
Akivaizdžiai matoma, jog didinamos vyriausybės išlaidos iššaukia didesnę
infliaciją. Pearsono koreliacijos koeficientas tarp šių dydžių yra 0.64,
o tai byloja apie vidutinio stiprumo teigiamą saryšį.

\begin{figure}

{\centering \includegraphics{Analize_files/figure-latex/unnamed-chunk-4-1} 

}

\caption{JAV vyriausybės išlaidos ir infliacija (1960-2019m).}\label{fig:unnamed-chunk-4}
\end{figure}

\hypertarget{mokesux10diai}{%
\subsection{Mokesčiai}\label{mokesux10diai}}

Antroji fiskalinė priemonė - mokesčių didinimas. Didesni mokesčiai
sumažina vartotojų disponuojamas pajamas, kas sumažina bendrą vartojimą
ir visuminę paklausą. Mokestinė valstybės sistema turėtų prisitaikyti
prie kylančių kainų, o gautos pajamos gali būti naudojamos biudžeto
perviršiui recesijos laikams kaupti. Augantys mokesčiai tuo pačiu galėtų
sugerti ir dėl augančios pinigų pasiūlos kylančią infliaciją. Tačiau
bendras visų mokesčių kėlimas nėra gera priemonė, apmokestinimas privalo
būti taiklus ir pasvertas. Pavyzdžiui, jei pakeliamas importo mokestis,
o šalies ekonomika ypatingai priklausoma nuo importuojamų prekių, tuomet
norimas infliacijos mažinimas bus paveiktas augančių importuotų prekių
kainų (Husain 2015). Taip pat mokestinės priemonės gali būti pritaikytos
ir siekiant suvaldyti pasiūlos šoko sukeltą infliaciją. Tokiu atveju
vyriausybės turi mažinti mokestinę naštą pramonės ir paslaugų sektoriaus
įmonėms taip siekiant sumažinti jų kaštus ir išlaikyti stabilias kainas,
tačiau tokios priemonės yra sunkiai prognozuojamos, kadangi įmonės gali
tikėtis, jog mokesčiai greitai vėl didės arba net ir sumažėjus įmonių
kaštams, jos gali ir toliau išlaikyti aukštas kainas ir net jas didinti,
ypač jei rinka monopolinė (Supel and others (1980)).

\hypertarget{monetarinux117s-priemonux117s}{%
\section{Monetarinės priemonės}\label{monetarinux117s-priemonux117s}}

Monetarinė politika - pagrindinė centrinių bankų vykdoma veikla.
Centriniai bankai reguliuoja pinigų pasiūlą, palūkanų normas
atsižvelgiant į ekonomikos ciklą. Infliacijos metu centrinis bankas
mažina pinigų pasiūlą ir didina palūkanų norma. Didesnė palūkanų norma
stabdo vartojimą, kadangi skolintis pasidaro brangiau, labiau apsimoka
taupyti, taip pat aukšta palūkanų norma padidina valiutos kursą, kas
savo ruožtu sumažina exportą ir didina importą. Palūkanų normos įtaka
infliacijai matoma grafikuose. Analizuojant pateiktas laiko eilutes,
galima įžvelgti, jog aukštesnė palūkanų norma dažnu atveju sąlygoja
krentančia infliaciją, tačiau kritimas dažniausiai pasireiškia
sekančiame laikotarpyje, tačiau egzistuoja ir laikotarpių, kai palūkanų
norma pakyla infliacijai pradėjus leistis, kas byloja apie per vėlai
naudojamas restrikcines monetarines priemones. (žr.1-ą pav). Centrinių
bankų ir palūkanų normos reakciją į infliaciją atspindi ryšys tarp
infliacijos ir palūkanų normos (žr.2-ą pav.). Pearsono koreliacijos
koeficientas yra 0.782, kas reiškia stiprią koreliaciją tarp šių
kintamųjų, todėl galime daryti išvadas, jog centriniai bankai greitai
adaptuoja palūkanų normą prie infliacijos lygio. Centrinis bankas taip
pat gali pasitelkti atvirosios rinkos operacijas ir rinkos dalyviams
pardavinėja vertybinius popierius taip sumažinant pinigų kiekį
apyvartoje. Komerciniai bankai, įsigiję vertybinių popierių, turės
mažiau skolinamų lėšų, o tai pabrangins skolinimąsį vartotojams bei
atsilieps bendram vartojimo lygiui. Kita priemonė - bankų privalomojo
rezervo reguliavimas. Infliaciniu laikotarpiu centrinis bankas didina
privalomų banko atsargų normą ir taip mažina kreditui galimų išduoti
lėšų skaičių. Tokia priemonė yra ypač veiksminga, kadangi rezervų
didinimas tiesiogiai ir labai greitai sumažina skolinamų lėšų skaičių.
Tačiau monetarinė politika turi ir trūkumų. Monetariniai instrumentai
nepasiekia norimų rezultatų, jei vartotojų lūkesčiai yra geri. Esant
teigiamiems lūkesčiams vartojimas bus didelis net ir didėjant palūkanų
normai. Kitas monetarinės politikos ribojimas yra laiko lagas (uždelstas
veikimas). Palūkanų normos poveikiui pasireikšti ir paveikti visuminę
paklausą gali prireikti iki 18 mėnesių, todėl efektui pasirodžius bendra
ekonomikos būklė jau gali reikalauti kitokių priemonių (Grant 2013).
Taip pat monetariniai įrankiai turi mažai įtakos pasiūlos šokui sukeltai
infliacijai.

\begin{figure}

{\centering \includegraphics{Analize_files/figure-latex/unnamed-chunk-5-1} 

}

\caption{JAV vyriausybės išlaidos ir infliacija (1960-2019m).}\label{fig:unnamed-chunk-5}
\end{figure}

\begin{figure}

{\centering \includegraphics{Analize_files/figure-latex/unnamed-chunk-6-1} 

}

\caption{JAV vyriausybės išlaidų ir infliacijos ryšys (1960-2019m).}\label{fig:unnamed-chunk-6}
\end{figure}

\hypertarget{kainux173-kontrolux117}{%
\section{Kainų kontrolė}\label{kainux173-kontrolux117}}

Kainų kontrolės mechanizmai naudoti dar antikoje, o ypač suklestėj
Antrojo pasaulinio karo metu ir po jo. Kainų kontrolės tikslas -
nustatyti maksimalias kainas gėrybėms taip užtikrinant stabilų kainų
lygį ir apsaugojant labiausiai pažeidžiamus visuomenės sluoksnius. Kainų
kontrolė yra geras įrankis, kai valstybėje, dėl gerybių trūkumo,
prasideda spekuliavimas ir suklesti juodoji rinka (Husain 2015).
J.M.Keynes'as kainų reguliavima laikė blogu įrankiu, kadangi jis trukdo
pasiekti pusiausvyros ekonomikos tašką, o laisvai svyruojančių kainų
sistema yra ypatingai svarbi kapitalizmui (Laguerodie and Vergara 2008).
Prieš kainų kontrolę pasisako ir daugelis kitų ekonomistų, kadangi kainų
kontrolė sutrukdo išteklių pasisiskirstymą, o kainų lubos sukelia prekių
deficitą (Rockoff 1992) (nustačius mažesnę už pusiausvyros kainą pakyla
paklausa, tačiau krenta pasiūla, todėl atsiranda prekės deficitas).

\newpage

\hypertarget{literatux16bros-sux105raux161as}{%
\section*{Literatūros sąrašas}\label{literatux16bros-sux105raux161as}}
\addcontentsline{toc}{section}{Literatūros sąrašas}

\hypertarget{refs}{}
\leavevmode\hypertarget{ref-ball1994determines}{}%
Ball, Laurence. 1994. ``What Determines the Sacrifice Ratio?'' In
\emph{Monetary Policy}, 155--93. The University of Chicago Press.

\leavevmode\hypertarget{ref-ball2017inflation}{}%
Ball, Robert James. 2017. \emph{Inflation and the Theory of Money}. Vol.
12. Transaction Publishers.

\leavevmode\hypertarget{ref-grant2013cambridge}{}%
Grant, Susan. 2013. \emph{Cambridge International as and a Level
Economics Revision Guide}. Cambridge University Press.

\leavevmode\hypertarget{ref-humphrey1981keynes}{}%
Humphrey, Thomas M. 1981. ``Keynes on Inflation.'' \emph{FRB Richmond
Economic Review} 67 (1): 3--13.

\leavevmode\hypertarget{ref-husainanti}{}%
Husain, SYED MUNIR. 2015. ``ANTI-Inflation Policy.''

\leavevmode\hypertarget{ref-ireland2014classical}{}%
Ireland, Peter. 2014. ``The Classical Theory of Inflation and Its Uses
Today.'' \emph{Retrieved on March} 23: 2019.

\leavevmode\hypertarget{ref-laguerodie2008theory}{}%
Laguerodie, Stephanie, and Francisco Vergara. 2008. ``The Theory of
Price Controls: John Kenneth Galbraith's Contribution.'' \emph{Review of
Political Economy} 20 (4): 569--93.

\leavevmode\hypertarget{ref-o2016keynes}{}%
O'Connell, Joan. 2016. ``On Keynes on Inflation and Unemployment.''
\emph{The European Journal of the History of Economic Thought} 23 (1):
82--101.

\leavevmode\hypertarget{ref-rockoff1992price}{}%
Rockoff, Hugh. 1992. \emph{Price Controls}. Elgar.

\leavevmode\hypertarget{ref-sargent1981stopping}{}%
Sargent, Thomas J, and others. 1981. ``Stopping Moderate Inflations: The
Methods of Poincare and Thatcher.''

\leavevmode\hypertarget{ref-supel1980supply}{}%
Supel, Thomas M, and others. 1980. ``Supply-Side Tax Cuts: Will They
Reduce Inflation?'' \emph{Quarterly Review}, no. Fall.

\leavevmode\hypertarget{ref-taylor1982union}{}%
Taylor, John B. 1982. ``Union Wage Settlements During a Disinflation.''
National Bureau of Economic Research Cambridge, Mass., USA.

\leavevmode\hypertarget{ref-weidenaar1979anti}{}%
Weidenaar, Dennis J. 1979. ``Anti-Inflationary Policies: Alternative
Approaches.''

\end{document}
